%%%%%%%%%%%%%%%%%%%%%%%%%%%%%%%%%%%%%%%%%%%%%%%%%%%%%%%%%%%%%%%%%
% Nome: Jullyana Vycas                                          %
% Disciplina: Recuperação da Informação                         %
%                                                               %
% Esse trabalho trata-se de um estudo do artigo Learning to     %
% Rank with Selection Bias in Personal Search, de autoria de    %
% Xuanhui Wang, Michael Bendersky, Donald Metzler e Marc Najork %
%%%%%%%%%%%%%%%%%%%%%%%%%%%%%%%%%%%%%%%%%%%%%%%%%%%%%%%%%%%%%%%%%


\documentclass{svproc}
%
% RECOMMENDED %%%%%%%%%%%%%%%%%%%%%%%%%%%%%%%%%%%%%%%%%%%%%%%%%%%
%

% Para língua portuguesa
\usepackage[utf8]{inputenc}

% Algoritmos
\usepackage{algorithm} 
\usepackage{algorithmic}
\renewcommand{\algorithmicrequire}{\textbf{Input:}}
\renewcommand{\algorithmicensure}{\textbf{Output:}}

% to typeset URLs, URIs, and DOIs
\usepackage{url}
\def\UrlFont{\rmfamily}

\begin{document}
\mainmatter              % start of a contribution
%
\title{Learning-to-Rank em Busca Personalizada}
%
\titlerunning{Learning-to-Rank em Busca Personalizada}  % abbreviated title (for running head)
%                                     also used for the TOC unless
%                                     \toctitle is used
%
\author{Jullyana Vycas}
%
\authorrunning{Jullyana Vycas} % abbreviated author list (for running head)
%
%%%% list of authors for the TOC (use if author list has to be modified)
\tocauthor{Jullyana Vycas}
%
\institute{Universidade Federal do Rio de Janeiro, Rio de Janeiro RJ, Brasil,\\
\email{jullytta.ufrj@gmail.com},\\ WWW home page:
\texttt{linkedin.com/in/jullytta}}

\maketitle              % typeset the title of the contribution

\begin{abstract}
Esse trabalho trata-se de um estudo das propostas do artigo
\textit{Learning to Rank with Selection Bias in Personal Search}, incluindo
uma implementação parcial do algoritmo Learning-To-Rank e experimentos
envolvendo os modelos vetorial e probabilístico.

\keywords{learning-to-rank, personal search, BM25, machine learning}
\end{abstract}
%
\section{Introdução}
%
Novas tecnologias ligadas a inteligência artificial e aprendizado de máquina estão cada vez mais presentes na área
de recuperação da informação, formando os chamados sistemas de recuperação inteligentes \cite{chen}. 
No lugar de especialistas no domínio que manualmente definem a relevância de um documento, entram algoritmos capazes 
de aprender a ranquear. Contudo, esse conhecimento não surge espontâneamente: é preciso obter um conjunto de dados
significativo para servir de exemplo para o algoritmo, ou seja, alimentá-lo.

Aprendemos em sala de aula que pode-se obter feedback dos usuários através de seus cliques. Se apresentamos um ranking 
como resultado de uma consulta, o documento que for acionado pelo usuário (ou seja, aquele que for clicado) provavelmente
é relevante. Portanto, podemos utilizar a avaliação do próprio usuário para descobrir empiricamente a relevância de um 
documento perante uma consulta. Evidentemente, precisamos de usuários fazendo essas buscas e verificando os documentos 
recuperados. Para grandes buscadores na Web, este não é um problema. O buscador Google, por exemplo, lida com
mais de 40.000 consultas por segundo \cite{google:search:statistics}. Cada uma dessas consultas gera dados preciosos
que servirão como exemplo para um algoritmo de aprendizado de máquina.

Contudo, o que acontece no caso em que o domínio é mais restrito? Para buscas dentro de documentos pessoais, sejam eles
e-mails ou arquivos, como descobrimos quais documentos são relevantes? Não podemos nos basear nas milhares de pesquisas
semelhantes que foram feitas anteriormente: os documentos recuperados são privados e diferentes para cada usuário.

Nas próximas seções, nos aprofundaremos no algoritmo Learning-To-Rank em sua forma utilizada para buscas na Web, fazendo 
experimentos sobre o mesmo, simulando uma busca sobre domínio particular. Em seguida, estudaremos as propostas apresentadas 
por Want et al. \cite{wang:bendersky:metzlet:najork} que adaptam o algoritmo para melhor se adequar a buscas personalizadas,
observando as diferentes tendências que podem influenciar a coleta dos dados.


\section{Learning-to-Rank}
Dada uma consulta $q$ e seu conjunto de resultados ${x_1,\dots,x_n}$, o objetivo do algoritmo
é encontrar uma função de ranqueamento $f(x)$ que minimize a perda total. A definição de perda
varia de acordo com a implementação. Neste trabalho, utilizamos uma função de perda par-a-par
simples, sugerida como modelo clássico pelo artigo estudado.

Para cada par $x_i$ e $x_j$ onde $x_i$ é mais relevante que $x_j$, temos:

\begin{equation}
perda_{par} = max(0, f(x_j) - f(x_i))^2
\end{equation}

A perda da função $f$ para com a consulta $q$, ou seja, $perda(f, q)$, é a soma da perda par-a-par para todo 
$x_i, x_j$ indicado acima. Dessa forma, estamos penalizando funções que possuem pares fora de ordem.
Além disso, a perda par-a-par é proporcional à distância entre os ranks.

Digamos que temos um conjunto finito de consultas que o usuário pode fazer, $Q$. 
Selecionamos um subconjunto $U$ de consultas, como amostra aleatória, para avaliar a função $f$.
Assim, a perda total é:

\begin{equation}
perda_{total}(f) = \frac{1}{|U|}\times \sum_{q \in U} perda(f, q)
\end{equation}

Ou seja, uma média aritmética das perdas para cada consulta $q$.

Note que para sermos capazes de calcular a perda apresentada, precisamos conhecer de antemão a ordem decrescente de 
relevância  dos documentos para cada uma das consultas. Vamos discutir como abordar esse problema na próxima seção. 
Por hora, vamos considerar que temos um ``ranking ideal'' contendo essa informação.

\begin{algorithm}
  \caption{Perda Total}
  \begin{algorithmic}[1]
    \REQUIRE conjunto de consultas U, ranking ideal para cada consulta R, função f(x)
    \ENSURE perda total de f
    \STATE $perda_{total} \Leftarrow 0$
    \FOR{$q$ em $U$}
      \STATE $perda_{consulta} \Leftarrow 0$
      \FORALL{$x_i, x_j$ onde $x_i$ aparece antes de $x_j$ em $R$}
	\STATE $perda_{par} \Leftarrow max(0, f(x_j) - f(x_i))^2$
	\STATE $perda_{consulta} \Leftarrow perda_{consulta} + perda_{par}$
      \ENDFOR
      \STATE $perda_{total} \Leftarrow perda_{total} + perda_{consulta}$
    \ENDFOR
    \STATE $perda_{total} \Leftarrow perda_{total} \times \frac{1}{|U|}$
  \end{algorithmic}
\end{algorithm}


% Bibliografia
\begin{thebibliography}{6}
%

\bibitem {wang:bendersky:metzlet:najork}
Wang, Xuanhui, Bendersky, Michael, Metzler, Donald, Najork, Marc: Learning to Rank with Selection Bias in Personal Search.
SIGIR (2016). \url{https://static.googleusercontent.com/media/research.google.com/en//pubs/archive/45286.pdf}

\bibitem {chen}
Chen, Hsinchun: Machine Learning for Information Retrieval: Neural Networks, Symbolic Learning, and Genetic Algorithms.
Journal of the American Society for Information Science (1995). \url{https://www.marilia.unesp.br/Home/Instituicao/Docentes/EdbertoFerneda/chen27.pdf}

\bibitem {google:search:statistics}
Internet Live Stats: Google Search Statistics.
Acesso em 13/06/2017. \url{http://www.internetlivestats.com/google-search-statistics/}


\end{thebibliography}
\end{document}
